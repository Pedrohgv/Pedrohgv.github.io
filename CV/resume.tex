%-------------------------
% Rover Resume - Base Template
% Link: https://github.com/subidit/rover-resume
%
% Shows code for various formatting options for different resume sections.
% Education and Projects have single-line headers; while Experience uses double-line.
% Some formatting codes are kept inline; consider \newcommand{cmd}{def}.
% Excludes hyperref and icons for readability; MVP version.
% Explore other templates for more options.
% Mix and match as desired. Be consistent with headers and sub-headers.
%------------------------

\documentclass[11pt]{article} % fontsize 10pt/11pt/12pt

\usepackage[margin=1in, a4paper]{geometry}
\setcounter{secnumdepth}{0} % remove section numbering
\usepackage{titlesec}
\titlespacing{\subsection}{0pt}{*0}{*0} % remove vertical spacing above and below
\titlespacing{\subsubsection}{0pt}{*0}{*0}
\titleformat{\section}{\large\bfseries\uppercase}{}{}{}[\titlerule]
\titleformat*{\subsubsection}{\large\itshape}
\usepackage{enumitem}
\setlist[itemize]{noitemsep,left=0pt .. \parindent}
\pagestyle{empty} % remove page number
\pdfgentounicode=1
\usepackage{hyperref}
\hypersetup{
    colorlinks=true,
    linkcolor=blue,
    filecolor=magenta,
    urlcolor=cyan,
    pdftitle={Resume - Pedro Venturott},
    pdfpagemode=FullScreen,
}


\begin{document}

\begin{center}
	\begin{minipage}{0.5\textwidth}
		{\Huge\bfseries
			PEDRO \\ VENTUROTT % Name Here
		} \\ \medskip
		Machine Learning Engineer % Title [optional]
	\end{minipage} \hfill
	\begin{minipage}{0.4\textwidth}
		\raggedleft
		Email: pedro.venturott@gmail.com \\
		Mobile: +55 27 99992-6511 \\
		Location: Brazil (UTC-03) \\
		\href{https://pedrohgv.github.io/}{Portfolio} \\
		\href{https://github.com/Pedrohgv}{GitHub} \\
		\href{https://pedro-venturott.medium.com/}{Medium} \\
		\href{https://www.linkedin.com/in/pedro-henrique-gomes-venturott/}{LinkedIn}
	\end{minipage}
\end{center}

\section{Summary}
%=================%
Self-taught Machine Learning Engineer with a passion for developing innovative ML solutions. Self-driven and independent learner with strong Python expertise. Particularly interested in applying ML technologies to Embedded Systems, Power Systems, Renewable Energy, NLP, and Computer Vision domains.

\section{Experience}
%=================%
\subsection{ReflexAI \hfill Remote}
\subsubsection{Machine Learning Engineer \hfill 03/2024 - Present}
\begin{itemize}
	\item Developed self-serve simulation product ecosystem, creating form-based interface for users to define simulation personas through structured questions, enabling automated simulation generation.
	\item Engineered LLM-powered natural language processing system using agentic approach that automatically populates persona creation forms from user prompts, exponentially increasing capacity to deliver simulations to customers.
	\item Implemented voice-based simulation capabilities by integrating third-party and self-hosted tools, designing and building the complete orchestration layer between Speech-to-Text (SST), Large Language Models (LLM), and Text-to-Speech (TTS) systems to deliver realistic voice interactions.
\end{itemize}

\subsection{Quilt.ai \hfill Remote}
\subsubsection{Machine Learning Engineer \hfill 11/2021 - 03/2024}
\begin{itemize}
	\item Spearheaded machine learning development initiatives, delivering advanced solutions in clustering, classification, vector embeddings, and text/image generation using Python ecosystem including NumPy, Scikit-Learn, PyTorch, HuggingFace, ChatGPT, and Llama.
	\item Architected and implemented comprehensive MLOps strategies to streamline ML solution deployment from development to production, optimizing models for efficiency and scalability through Docker, Kubernetes, AWS SageMaker, GCP Cloud Run, FastAPI, ONNX optimization, and model quantization.
\end{itemize}

\subsection{Kerberos.io \hfill Remote}
\subsubsection{Part-Time Machine Learning Engineer \hfill 06/2021 - 11/2021}
\begin{itemize}
	\item Developed NVIDIA GPU-accelerated Kubernetes infrastructure for computer vision workloads, implementing end-to-end object detection pipeline with reusable Kubeflow components for model training, inference, and monitoring using Prometheus and Grafana.
\end{itemize}

\subsection{Freelancing \hfill Remote}
\subsubsection{Machine Learning Engineer \hfill 02/2021 - 06/2021}
\begin{itemize}
	\item Delivered client-specific ML solutions across multiple projects, specializing in data preprocessing, time-series prediction, and recurrent neural network implementations.
\end{itemize}

\section{Education}
%=================%
\subsection{Federal University of Espirito Santo, Brazil \\ {\normalfont\itshape Bachelor of Science in Electrical Engineering} \hfill 08/2010 - 08/2017}

\vspace{0.3cm} % Adding space between education subsections

\subsection{Stanford University on Coursera \\ {\normalfont\itshape Machine Learning Course and Deep Learning Specialization} \hfill 01/2019 - 03/2019}

\vspace{0.3cm} % Adding space after Education section

\section{Technical Skills}
%===========================%
\begin{description}[itemsep=0pt]
	\item[MLOps:] SageMaker, Docker, Kubernetes, Kubeflow, Kafka, GCP, Prefect, Cloud Run, ONNX
	\item[Python:] NumPy, Pandas, Scikit-Learn, StatsModels, LifeLines, sktime, TensorFlow, PyTorch, Matplotlib, Plotly, Flask, FastAPI
	\item[Machine Learning:] Logistic/Linear Regression, SVM, Random Forest, Naive Bayes, KNN, K-Means, Time-Series ML, Deep Learning, NLP, Computer Vision
	\item[AI/LLM:] Prompt Engineering, Text/Image Generation, Conversational AI, AI Agents
	\item[Data Engineering:] Data Processing and Cleaning, Pipeline Development
	\item[Tools:] Git, Jira
\end{description}

\section{Languages}
%=================%
\begin{description}[itemsep=0pt]
	\item[English] Professional Proficiency
	\item[Portuguese] Native
\end{description}

\section{Projects}
%=================
\subsection{NVIDIA GPU Kubernetes for Computer Vision $|$ \normalfont\textit{\href{https://github.com/Pedrohgv/nvidia-gpu-kubernetes}{GitHub}}}
\begin{itemize}
  \item Developed comprehensive Kubernetes infrastructure leveraging NVIDIA GPU acceleration for computer vision workloads, implementing end-to-end ML pipelines with Kubeflow for scalable object detection in surveillance systems.
\end{itemize}

\subsection{Vehicle Type Classification Using Simulated Trajectory Data $|$ \normalfont\textit{\href{https://github.com/Pedrohgv/standard-traffic-data/blob/main/knowledge/Vehicle_Type_Prediction.ipynb}{GitHub}/\href{https://towardsdatascience.com/vehicle-type-prediction-using-simulated-trajectory-data-59a522e30003/}{Medium}}}
\begin{itemize}
  \item Developed machine learning model to classify vehicle types from simulated trajectory data, achieving an F1-Score of 0.87 through feature engineering and model optimization.
\end{itemize}

\subsection{Music Genre Classification Using Waveform Features $|$ \normalfont\textit{\href{https://github.com/Pedrohgv/Music_Genre_Classification/blob/master/notebook.ipynb}{GitHub}/\href{https://towardsdatascience.com/predicting-music-genres-using-waveform-features-5080e788eb64/}{Medium}}}
\begin{itemize}
  \item Built classification system to categorize music into 10 genres using waveform feature extraction techniques, implementing signal processing algorithms to achieve 68.5\% accuracy.
\end{itemize}

\subsection{Character-level Short Text Generator $|$ \normalfont\textit{\href{https://github.com/Pedrohgv/Star-Wars-Text-Generation}{GitHub}/\href{https://towardsdatascience.com/generating-short-star-wars-text-with-lstms-c7dc65e252c4/}{Medium}}}
\begin{itemize}
  \item Created deep learning-based text generation system by scraping Star Wars Wikipedia content and training character-level recurrent neural network to generate contextually relevant short text.
\end{itemize}

\end{document}
